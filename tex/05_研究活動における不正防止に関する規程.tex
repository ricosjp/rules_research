\documentclass[10pt,a4paper,uplatex]{jsarticle}
\usepackage{bm}
\usepackage{graphicx}
\usepackage[truedimen,left=25truemm,right=25truemm,top=25truemm,bottom=25truemm]{geometry}
\usepackage{array}
\usepackage{titlesec}
\usepackage{jpdoc}
\usepackage[nomarkers]{endfloat}

\titleformat*{\section}{\large\bfseries}
\def\title{株式会社科学計算総合研究所\\研究活動における不正防止に関する規程}

\def\alias#1{\十干{#1}}

\begin{document}
	\newpage
{\centering \Large\bf \title  \vskip 0em}
\vskip 2em

\article{目的}
この規程は、「研究活動における不正行為への対応等に関するガイドライン」(平成26年8月26日 文部科学大臣決定)に基づき、株式会社科学計算総合研究所(以下、「当会社」とする。)における研究活動にかかわる不正行為(以下「不正」という。)を防止し、不正行為が行われ、又はその恐れがある場合に、適正かつ迅速に対応するために必要な事項を定めることにより、研究倫理の保持及び向上を図ることを目的とする。

\article{定義}
この規程において「研究上の不正行為」とは、研究活動に伴い外部に発表した研究成果(論文、学会発表、報告書等)におけるねつ造、改ざん及び盗用をいう。なお、重大な過失がある場合を含む。
\begin{enumerate}
	\item「ねつ造」とは、存在しないデータ等を用い、調査研究結果等を作成することをいう。
	\item「改ざん」とは、調査研究資料・過程を変更する操作を行い、データ、調査研究活動によって得られた結果を真正でないものに加工することをいう。
	\item「盗用」とは、他の研究者のアイデア、分析・解析方法、データ、研究結果、用語を、当該研究者の了解又は適切な表示なく流用することをいう。
\end{enumerate}

\article{責務}
当会社は、研究倫理および規程に則り、研究者自身の自律的な取り組みを基本としつつ、組織として不正行為に適切に対応する仕組みを整備し、実効性のある取り組みを推進するものとする。

\article{体制}
研究所全体を統括し、不正の防止及び適正な運営管理について最終責任を負う者(最高管理責任者)を代表取締役と定める。
\term 公的な研究費に関わる運営・管理について実質的な責任と権限を持つ者(コンプライアンス推進責任者)を基盤研究部長と定める。
\term 研究者等に対する研究倫理教育について実質的な責任と権限を持つ者(研究倫理教育責任者)を基盤研究部長と定める。
\term 総務部は、この規程に基づき、事務を補佐する。

\article{コンプライアンス教育・研究倫理教育受講の義務}
当会社の研究活動に関わる者は、コンプライアンス教育・研究倫理教育を受講するものとする。

\article{データの保存及び開示}
当会社における研究データとは、調査研究活動にともない発生または使用する文書、数値データ、画像等のうち、外部に発表した研究成果(論文、学会発表、報告書等)にかかわるもので、発表した者が当該調査研究活動の正当性等を説明するときに必要となるものをさす。
\term 研究データのうち、文書、数値データ、画像等の保存期間は、原則として、論文や報告等の成果発表後10年間とする。電子データについては、作成者、作成日時及び属性等の整備と適切なバックアップ等の作成により再利用可能な形で保存すること。なお、その他紙媒体の資料等についても、少なくとも10年の保存が望ましいが、保管スペースの制約など止むを得ない事情がある場合には、合理的な説明がつく範囲で廃棄することも可能とする。 
\term 本規程は、最低限保存する期間を示すものであり、当該論文等が世界的に極めて顕著な研究成果である場合や長く保存することが可能である場合等については、本規程に定める保存期間にかかわらず、必要に応じ、保存期間を延長できるものとする。
\term 本規程に定める保存期間の終了以前に、合理的な理由なく廃棄した場合等は、不正行為とみなされる場合がある。
\term 個々の研究者が実践すべき研究データの保存について、基盤研究部長は、研究活動の健全性が担保されるよう、適切な教育・指導と環境整備に努めるものとする。また、保存すべき研究データの担当研究者および所在、保存期間、方法等について把握し、管理を行うこととする。
\term 研究責任者は、研究者の転出や退職に際して、当該研究者の研究活動に関わる保存すべき研究データの保存場所及び後日確認が必要となった場合の連絡方法等について、当該研究者と確認した内容を記載したものを保管し、追跡可能としておくこと。また、必要に応じ、研究データ等のバックアップを保管するなどの措置を講ずること。
\term 研究者及び研究責任者は、論文等の形で発表した研究成果について、求めに応じ、研究活動の適正性について科学的根拠をもって説明するとともに、必要に応じ、研究データ等を開示しなければならない。なお、転出や退職後もその責務を負うものとする。
\term 第2項及び第3項の規定にかかわらず、特定の研究プロジェクトに関して研究資料等もしくは成果物等の取扱いについて資金提供機関もしくは研究資
料等の提供機関との取り決めや契約等がある場合にはそれに従う。

\article{通報・相談窓口の設置}
不正に関する通報および相談を受け付ける窓口は、総務部がその任にあたる。
\term 総務部は、通報・相談が、書面、電話、電子メール、面談等の適宜の方法により行えるように窓口の体制を整備する。
\term 当会社は、通報・相談の内容および通報者の秘密を守るための適切な方法を講じ、これを保護しなければならない。

\article{通報の受理等}
通報は、第7条の通報窓口に対し直接行うものとする。
\term 総務部は通報を受けたときは、速やかに、最高管理責任者に報告するものとする。
\term 通報は、次に掲げるすべての事項が明示されている場合のみを受理することとし、当該通報者に対して受理したことを通知する。
\begin{enumerate}
	\item 不正を行ったとする役職員の氏名(以下「被通報者」という。)
	\item 不正の態様
	\item 不正と判断した合理的な理由
\end{enumerate}
\term 通報は、原則として顕名によるもののみを受理するものとする。ただし、匿名によるものであっても、その内容に応じ、顕名の場合に準じた取扱いをすることができる。
\term 告発の意思を明示しない相談の場合には、相談者に対し、告発の意思の有無を確認する。
\term 前項において告発の意思が確認されない場合にも、通報を受理した場合に準じた取扱いをすることができる。
\term 他の機関から告発が回付または通知された場合は、当会社に通報があったものとして当該事案を取り扱う。

\article{学会等・報道による指摘等}
学会等もしくは報道により不正が指摘された場合、又は、不正の疑いがインターネット上に掲載されていること(ただし、当該事案の内容が明示され、かつ不正とする科学的な合理性のある理由が示されている場合に限る)を通報窓口が確認した場合には、通報があった場合に準じて取り扱うことができる。 

\article{予備調査}
最高管理責任者は、通報を受理した時は、不正に関して本調査が必要かどうかを検討するため、予備調査を行う。
\term 最高管理責任者は、予備調査を行う場合、役職員に対しそれらが保有する資料の保全等を命ずることができる。
\term 最高管理責任者は、予備調査を行うことを被通報者に通知する。
\term 最高管理責任者は、通報を受理した日から30日以内に予備調査を終了し、その結果を通報者および被通報者に開示する。

\article{未発生の不正についての通報・相談に対する予備調査}
最高管理責任者は、不正が行われようとしている、又は不正を求められているという通報・相談があった場合において、予備調査の結果、不正があったことが疎明された場合において必要があると認められるときは、被通報者に警告を行うことができる。ただし、被通報者が当会社の役職員でない場合は、被通報者の所属する研究機関に事案を回付することができる。役職員でない被通報者に警告を行った場合には、被通報者の所属する研究機関に警告の内容等について通知する。

\article{調査委員会の設置等}
最高管理責任者は、予備調査の結果、本調査が必要であると判断した場合、調査委員会を設置する。
\term 最高管理責任者は、調査委員会の委員を、当会社の役職員でない有識者から任命または委嘱する。このとき、委員の半数以上が外部有識者で構成され、すべての委員が通報者および被通報者と直接の利害関係を有しない者で構成されるようにしなければならない。
\term 調査委員会を設置した場合には、調査委員の氏名及び所属を、通報者及び被通報者に通知する。
\term 通報者および被通報者は、前項の通知があった日から7日以内に、調査委員について異議を申し立てることができる。
\term 最高管理責任者は、前項の異議申立てがあった場合、内容を審査し、その内容が妥当であると判断したときは、当該申立てに係る委員を交代させるとともに、その旨を通報者及び被通報者に通知する。

\article{本調査の通知}
最高管理責任者は、本調査を行うことを決定した場合は、当該事案ににかかる研究の費用を助成した機関(以下「配分機関」という。)及び当該事案にかかる研究を委託し、もしくはその費用を助成した省庁(以下「関係省庁」という。)に対し、本調査を行うことを報告する。
\term 当該事案の通報者及び被通報者に対し、本調査を行うことを通知するとともに、本調査への協力を求める。
\term 本調査を行わないことを決定した場合には、通報者に通知し、当該事案に関わる機関及び通報者の求めがあった場合において、最高管理責任者が必要と認めたときは予備調査に関わる資料等を開示する。

\article{証拠となる資料等の保全}
本調査の実施にあたっては、通報された事案に関わる調査研究活動に関し、証拠となるような資料等を保全する。
\term 役職員が当会社とは別の研究機関と共同して調査、研究を行う場合において(以下、この研究機関を「共同研究機関」という。)、その共同研究機関の調査機関から要請があった場合は、前項に準じるものとする。

\article{秘密とすべき情報の管理}
本調査の実施にあたって、調査対象における公表前のデータ、研究上または技術上秘密とすべき情報が、調査の遂行上必要な範囲外に漏えいすることがないよう十分配慮する。

\article{被通報者の調査研究活動の制限}
最高管理責任者は、本調査の結果が出る前であっても、被通報者に対し、調査対象となった事案と同じ制度による研究費の使用を一時停止させることができる。
\term 第14条の資料等の保全に影響しない限り、被通報者の調査研究活動を妨げない。

\article {本調査の実施}
調査委員会は、予備調査が終了した日から30日以内に本調査を開始する。
\term 調査委員会は、本調査を開始した日から原則として150日以内に、不正の有無の認定とその具体的内容及びその根拠とした調査の内容をまとめた報告書を最高管理責任者に対し提出する。
\term 調査委員会は、調査にあたって、被通報者の弁明を聴取する。
\term 調査委員会は、通報にかかる事実が不正に当たらないにもかかわらず通報がなされた場合において、通報者に故意または重過失がある旨の認定を行うとき(以下、かかる認定がなされた場合を「悪意に基づくものと認定された場合」という。)は、あらかじめ通報者の弁明を聴取する(以下かかる認定がなされた通報者を「悪意の通報者」という。)。
\term 被通報者の自認を唯一の証拠として不正と認定することはできない。

\article{最高管理責任者への報告}
調査委員会は、第17条2項の報告書をもって、ただちに最高管理責任者に報告する。
\term 調査委員会は、調査期間中であっても、最高管理責任者の求めがあった場合、中間報告書を提出する。

\article{本調査結果の通知と報告}
最高管理責任者は、前条第1項の報告書の提出があった場合には、速やかに、通報者および被通報者に通知するとともに、当該配分機関および関係省庁に報告する。

\article{不服申立て}
前条により通知された調査結果において、不正を行ったと認定された被通報者または悪意の通報者は、前条による通知の日から10日以内にその調査結果に不服を申し立てることができる。
\term 不服申立てがあった場合は、すみやかに関係者に通知するとともに、配分機関及び関係省庁に報告する。

\article{再調査}
前条の不服申立ての審査は、当該事案の本調査を行った調査委員会が行う。ただし、被認定者の不服申立ての内容について新たな専門性を要する判断が必要となる場合は、最高管理責任者は、調査委員の交代もしくは追加または調査委員会に代えて他の者に審査を行わせることができる。
\term 調査委員会は、前項により当該事案の再調査を行うか否かを審理し、不服申立がなされた日から10日以内に最高管理責任者に対しその結果を報告するものとする。
\term 最高管理責任者が再調査の実施を決定した場合には、決定した日から、調査委員会は原則として10日以内に本調査を開始する。
\term 再調査を行う場合には、再調査を開始した日から原則として50日以内に、調査報告書を作成し最高管理責任者に提出する。
\term 最高管理責任者は、不服申立をした者に対し、再調査の結果を通知するとともに、当該不服申立ての経緯とその調査結果等を、配分機関及び関係省庁に報告する。

\article{調査結果の公表}
最高管理責任者は、不正が行われたと認定された場合には、速やかに調査結果を公表するものとする。
\term 公表する調査結果の内容(項目等)は、特段の事情がない限り、不正に関与した者の氏名・所属、不正の内容、処分の内容、調査の方法等とする。
\term 不正が行われなかったと認定された場合には、原則として調査結果は公表しないものとする。但し、調査事案が外部に漏えいしていた場合、不正でないが当該事案にかかる調査研究において看過し得ない誤りが判明した場合、その他最高管理責任者が相当と認める場合、必要な事項を公表することができる。
\term 通報が悪意に基づくものと認定されたときには、調査結果を公表する。

\article{不正に対する措置}
最高管理責任者は、不正が行われたと認定された場合、または通報が悪意に基づくものと認定された場合は、当会社就業規則第47条に基づいて適切な処置を講じる。

\article{調査への協力}
役職員等は、この規程に基づく調査等に協力しなければならない。

\article{秘密の保持}
役職員および調査委員に委嘱された有識者は、この規程に規定する調査等に関して知ることができた情報を調査関係者以外に漏らしてはならない。

\article{調査等の事務に携わる者の制限}
不正行為の事案の事務に携わる者は、自らが関係すると考えられる事案の処理に関与してはならない。
\\

\subparagraph{附則}
この規程は、令和2年10月5日から施行する。

\end{document}
