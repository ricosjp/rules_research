\documentclass[10pt,a4paper,uplatex]{jsarticle}
\usepackage{bm}
\usepackage{graphicx}
\usepackage[truedimen,left=25truemm,right=25truemm,top=25truemm,bottom=25truemm]{geometry}
\usepackage{array}
\usepackage{titlesec}
\usepackage{jpdoc}
\usepackage[nomarkers]{endfloat}

\titleformat*{\section}{\large\bfseries}
\def\title{株式会社科学計算総合研究所\\公的研究費等の運営・管理体制に関する規程}

\def\alias#1{\十干{#1}}

\begin{document}
	\newpage
{\centering \Large\bf \title  \vskip 0em}
\vskip 2em

\article{目的}
この規程は、「研究機関における公的研究費の管理・監査のガイドライン(実施基準)」(平成19年2月15日(令和3年2月1日改正)文部科学大臣決定)に基づき、株式会社科学計算総合研究所(以下、「当会社」とする)における公的研究費の取扱いに関して、適正な運営・管理及び不正使用防止について必要な事項を定めることを目的とする。

\article{最高管理責任者}
当会社における公的研究費等による研究の取扱いに係る最高管理責任者を、代表取締役とする。
\term 最高管理責任者は、不正防止対策の基本方針を策定する。重要事項を審議する場においては、審議を主導するとともに実施状況や効果等について統括管理責任者、コンプライアンス推進責任者及び内部監査部門等と議論を深めた上で、策定する。
\term 最高管理責任者は、公的研究費等による研究の取扱いについて最終的な責任を負う。
\term 最高管理責任者は、統括管理責任者、コンプライアンス推進責任者が責任を持って公的研究費等による研究に係る運営及び管理を円滑に行えるよう、適切に調整及び支援を行う。
\term 最高管理責任者は、様々な啓発活動を定期的に行い、当会社全体の不正防止に向けた取組を促すなど、公的研究費等の運営・管理に関わる全ての構成員(以下、「職員等」とする)の意識の向上と浸透を図る。
\term 最高管理責任者は、不正な取引に関与した調達等を行う業者への取引停止等の処分方針を機関として定め、業者に対し周知徹底するものとする。

\article{統括管理責任者}
当会社における公的研究費等による研究の取扱いに係る統括管理責任者を、基盤研究部長とする。
\term 統括管理責任者は最高管理責任者を補佐し、公的研究費等による研究の実質的処理に関する責任を負う。
\term 統括管理責任者は、不正防止計画をはじめとする当会社全体の具体的な対策を策定・実施し、実施状況を確認するとともに、実施状況を最高管理責任者に報告する。不正防止計画は、内部監査部門等と意見交換を行い、モニタリングの結果やリスクが顕在化したケースの状況等を活用し、定期的に点検し、必要に応じて見直しを行う。
\term 統括管理責任者は、コンプライアンス教育や啓発活動の実施計画を策定し、公的研究費等の運営・管理に関わる職員等の意識の向上と浸透を促し、組織全体で不正を防止する風土を形成するための全体的な取り組みを実施し、実施状況を確認するとともに、実施状況を最高管理責任者に報告する。

\article{コンプライアンス推進責任者}
当会社における公的研究費等による研究の取扱いに係るコンプライアンス推進責任者を、基盤研究部長とする。
\term コンプライアンス推進責任者は、不正防止対策を実施し、実施状況を確認するとともに、実施状況を最高管理責任者に報告する。
\term コンプライアンス推進責任者は、公的研究費等の運営・管理に関わる職員等に対しコンプライアンス教育を実施し、受講状況を管理監督する。
\term コンプライアンス推進責任者は、自己の管理監督又は指導する部等において、公的研究費等の運営・管理に関わる職員等が、適切に公的研究費等の管理・執行を行っているか等をモニタリングし、必要
に応じて改善を指導する。
\term コンプライアンス推進責任者は、自己の管理監督又は指導する部等において、様々な啓発活動を定期的に行い、不正防止に向けた取組を促すなど、公的研究費等の運営・管理に関わる職員等の意識の向上と浸透を図る。
\term コンプライアンス推進責任者は、コンプライアンス教育の内容を定期的に点検し、必要に応じて見直しを行う。

\article{研究倫理教育責任者}
当会社における公的研究費等による研究の取扱いに係る研究倫理教育責任者を、基盤研究部長とする。
\term 研究倫理教育責任者は、公的研究費等による研究の運営・管理に関わる職員等に対し研究倫理教育を実施し、受講状況を管理監督する。
\term 研究倫理教育責任者は、所属する研究者等が適切に研究活動を行っているか管理・及び必要に応じた改善指導する責任を負う。

\article{コンプライアンス教育・研究倫理教育}
公的研究費等による研究の運営・管理に関わる職員等は、次の各号に掲げる事項を行動規準として、公的研究費等による研究の運営・管理に関わる活動を行わなければならない。
\begin{enumerate}
	\item 公的研究費の不正使用や研究活動における不正行為を行わないこと。
	\item 公的研究費の不正使用や研究活動における不正行為に加担しないこと。
	\item 周囲の者に対し、公的研究費の不正使用や研究活動における不正行為を行わせないこと。
\end{enumerate}
\term 公的研究費等による研究の運営・管理に関わる職員等は、コンプライアンス教育・研究倫理教育を受講しなければならない。
\term 公的研究費等による研究の運営・管理に関わる職員等は、次の事項を含む誓約書(別途様式に定める)を当会社に提出しなければならない。
\begin{enumerate}
	\item 当会社及び公的研究費の配分機関の規則等を遵守すること
	\item 公的研究費の不正使用や研究活動における不正行為を行わないこと
	\item 規則等に違反して、不正を行った場合は、機関や配分機関の処分及び法的な責任を負担すること 
\end{enumerate}
\term 前項に係る誓約書を提出しない者は、公的研究費等の運営・管理、研究活動に係ることができないものとする。

\article{不正要因の把握及び不正防止計画の策定}
公的研究費等の不正使用を未然に防止するため、不正を発生させる要因を把握し、具体的な不正防止計画を策定し、不正の発生防止に努める。
\term この策定に係る事務は、総務部が兼務する。
\term 総務部は、機関全体の状況を体系的に把握するため、統括管理責任者に意見を聞いた後、最高管理責任者の承認を得て、不正防止計画を策定する。

\article{公的研究費の管理}
物品・役務の購入を希望するにあたり、不正を抑止し透明性や客観性を担保するため、上長の決裁の後、最高管理責任者の決裁によって購入を決定する。また、決裁前に購入する際に利用する公的研究費の支出財源を特定しておくこととする。
\term 物品の発注は、研究者の依頼に基づき総務部が行う。研究者本人は発注を行わない。 
\term 物品の検収は、総務部が行う。発注データ(仕様書や契約書等)との照合が必要な場合は、担当研究者が検収時に立ち会い、確認すること。据え付け調整等の設置作業を伴う場合は、設置後の現場で納品を確認すること。
\term 特殊な役務(データベース・プログラム・デジタルコンテンツ作成、機器の保守・点
検など)に関する検収を行う場合、成果物・報告書等の役務履行が確認できる書類を受領すること。また、これらの役務に係る知識を有する発注者以外のものも検収に参加すること。成果物のない機器の保守・点検等の場合には、総務部の担当者が検収時に立ち会い、役務履行を確認すること。
\term 研究者等が国内外への出張する場合に支給する旅費については、総務部から最高管理責任者に決裁伺いをとる。用務終了後に、出張報告書、領収書及び航空券半券等により総務部が事実確認を行う。
\term 公的研究費を雇用財源とした非常勤雇用者に係る雇用の判断及び勤務状況等の雇用管理は、総務部が行う。非常勤雇用者の雇用管理として、研究責任者と総務部が定期的な面談や勤務条件の説明、及び定期的な出勤簿(勤務内容を含む)の確認を行う。
\term 換金性の高い物品は、納品時に管理番号を付与し、対象物品に財源となった競争的資金の名称及び管理番号を明記する。また、物品の所在が追跡できるように台帳にて管理すること。

\article{相談窓口}
当会社の公的研究費等による研究の運営・管理に関する統一的な運用の相談窓口は、総務部とする。

\article{通報窓口}
当会社の公的研究費等による研究の不正行為に関する通報窓口は、総務部とする。
\term 告発等(報道や会計検査院等の外部機関からの指摘を含む)を受け付けた場合は、告発等の受付から30日以内に、告発等の内容の合理性を確認し調査の要否及び調査委員会の設置要否を判断するととともに、当該調査の要否を配分機関に報告する。

\article{不正を行った業者への処分}
「不正な取引に関与した業者への取引停止の処分方針」の定めた内容に従って処分する。

\article{調査委員会}
調査が必要と判断された場合は、調査委員会を設置し、調査(不正の有無及び不正の内容、関与した者及びその関与の程度、不正使用の相当額等についての調査)を実施する。
\term 不正に係る調査体制については、公正かつ透明性の確保の観点から、当会社に属さない第三者(弁護士、公認会計士等)を含む調査委員会を設置する。
\term 第三者の調査委員は、当会社及び告発者、被告発者と直接の利害関係を有しない者でなければならない。
\term 当会社は必要に応じて、非告発者等の調査対象となっている者に対し、調査対象制度の研究費の使用停止を命ずる。
\term 調査委員会は、不正の有無及び不正の内容、関与した者及びその関与の程度、不正使用の相当額等について認定する。

\article{調査の報告}
当会社は、調査の実施に際し、調査方針、調査対象及び方法等について配分機関に報告、協議しなければならない。
\term 告発等の受付から210日以内に、調査結果、不正発生要因、不正に関与した者が関わる他の競争的資金等における管理・監査体制の状況、再発防止計画等を含む最終報告書を配分機関に提出する。期限までに調査が完了しない場合であっても、調査の中間報告を配分機関に提出する。
\term 調査の過程であっても、不正の事実が一部でも確認された場合には、速やかに認定し、配分機関に報告する。
\term 配分機関の求めに応じ、調査の終了前であっても、調査の進捗状況報告及び調査の中間報告を当該分配機関に提出する。
\term 調査に支障がある等、正当な事由がある場合を除き、当該事案に係る資料の提出又は閲覧、現地調査に応じることとする。

\article{対応措置}
最高管理責任者は、不正の事実があると認めたときはその者に対して当会社の職業規則等に基づき処分を行うものとする。
\term 最高管理責任者は、調査の結果、その者に対して不正の事実がないと認めたときは、不利益発生要因防止策及び名誉回復に係る措置を講ずるものとする。
\term 調査の結果、通報者による研究妨害その他の作為的な行為であることが明らかとなった場合には、最高管理責任者は当該通報者に対し、関係法令等に基づき必要な措置を講ずるものとする。\\
\\
\subparagraph{附則}
この規程は、令和3年9月1日から施行する。

\end{document}
