\documentclass[10pt,a4paper]{jsarticle}
\usepackage{bm}
\usepackage{graphicx}
\usepackage[truedimen,left=25truemm,right=25truemm,top=25truemm,bottom=25truemm]{geometry}
\usepackage{array}
\usepackage{titlesec}
\usepackage{jpdoc}
\usepackage[nomarkers]{endfloat}

\titleformat*{\section}{\large\bfseries}
\def\title{株式会社科学計算総合研究所\\公的研究費等の不正防止に関する基本方針}

\def\alias#1{\十干{#1}}

\begin{document}
	\newpage
{\centering \Large\bf \title  \vskip 0em}
\vskip 2em

\rightline{令和2年10月5日制定}

\article{目的}
株式会社科学計算総合研究所(以下、「当会社」とする。)は、「研究機関における公的研究費の管理・監査のガイドライン(実施基準)」(平成19年2月15日(平成26年2月18日改正)文部科学大臣決定)に基づき、公的研究費等の不正使用を防止し、適正な運営及び管理を行うための基本方針を定める。

\article{責任体制の明確化}
当会社における公的研究費を適正に運営及び管理するために、「最高管理責任者」、「統括管理責任者」、「コンプライアンス推進責任者」及び「研究倫理教育責任者」を置き、各責任者が不正防止対策に関して責任をもち、積極的に推進するとともに、その役割・責任の所在・範囲と権限を明確化し責任体系を当会社内外に周知・公表する。

\article{ルールの明確化・統一化}
公的研究費等の使用及び事務手続きに関するルールについて、明確かつ統一的な運用を図るとともに、公的研究費等の運営及び管理に関わる全ての職員に周知を図る。

\article{職務権限の明確化}
公的研究費等の事務処理に関する職員の権限と責任について定め、職務権限に応じた明確な決裁手続きを定める。

\article{関係者の意識向上}
公的研究費等の運営及び管理に関わる全ての職員に対して、当会社の不正対策に関する方針やルール等に関するコンプライアンス教育を実施し、受講者の受講状況及び理解度を把握するとともに、関係する規則等を遵守する旨の誓約書の提出を求める。

\article{告発等に関する手続きの明確化}
公的研究費の不正使用に係る告発等の取扱い、不正使用に係る調査及び不正行為等懲戒に関する規程または細則を整備し、その運用の透明化に努める。

\article{不正防止計画の策定及び実施}
公的研究費等の不正使用を未然に防止するため、不正を発生させる要因を把握し、具体的な不正防止計画を策定し、不正の発生防止に努める。

\article{研究費の適正な運営及び管理活動}
公的研究費等の適正な運営及び管理活動を図るため、不正防止計画を踏まえた適切な予算執行を行う。また、業者との癒着の発生を防止するとともに、不正につながりうる問題が捉えられるよう、実効性のあるチェックが可能なシステムを策定し管理する。

\article{情報発信及び共有化の推進}
当会社における公的研究費等の不正防止に向けた取り組みについて、方針及び手続等を情報発信するとともに、内部においても情報共有する。

\article{モニタリング及び監査の在り方}
公的研究費等の適正なる管理のため、当会社全体の視点から、実効性のあるモニタリング及び監査制度を準備し、実施する。

\end{document}
