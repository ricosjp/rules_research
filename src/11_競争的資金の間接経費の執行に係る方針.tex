\documentclass[10pt,a4paper,uplatex]{jsarticle}
\usepackage{bm}
\usepackage{graphicx}
\usepackage[truedimen,left=25truemm,right=25truemm,top=25truemm,bottom=25truemm]{geometry}
\usepackage{array}
\usepackage{titlesec}
\usepackage{jpdoc}
\usepackage[nomarkers]{endfloat}

\titleformat*{\section}{\large\bfseries}
\def\title{競争的資金の間接経費の執行に係る方針}

\def\alias#1{\十干{#1}}

\begin{document}
	\newpage
{\centering \Large\bf \title  \vskip 0em}
\vskip 2em

\rightline{令和5年7月1日制定}

\article{目的}
株式会社RICOS(以下、「当会社」とする。)は、「競争的資金の間接経費の執行に係る共通指針」(平成13年4月20日(平成17年3月23日改正、平成21年3月27日改正、平成26年5月29日改正)競争的資金に関する関係府省連絡会申し合わせ)(以下、「共通指針」とする。)に基づき、競争的資金の間接経費を効果的かつ効率的に活用及び円滑に運用するための方針を定める。

\article{定義}
「配分機関」とは、競争的資金の制度を運営し、競争的資金を研究機関又は研究者に配分する機関をいう。 
\term 「直接経費」とは、競争的資金により行われる研究を実施するために、研究に直接的に必要なものに対し、競争的資金を獲得した研究機関又は研究者が使用する経費をいう。
\term 「間接経費」とは、直接経費に対して一定比率で手当され、競争的資金による研究の実施に伴う研究機関の管理等に必要な経費として、当会社が使用する経費をいう。

\article{間接経費の使途}
間接経費は、競争的資金を獲得した研究者の研究開発環境の改善や研究機関全体の機能の向上に活用する。
\term 間接経費の使途については、共通指針に準じるものとする(具体的な主な使途は下記)。下記以外であっても、競争的資金を獲得した研究者の研究開発環境の改善や研究機関全体の機能の向上に活用するために必要となる経費などで、最高管理責任者が必要な経費と判断した場合、執行することは可能とする。なお、直接経費として充当すべきものは対象外とする。
\begin{enumerate}
	\item 管理部門に係る経費
		\begin{enumerate}
			\item 管理施設・設備の整備、維持及び運営経費 
			\item 管理事務の必要経費(備品購入費、消耗品費、機器借料、雑役務費、人件費、通信運搬費、謝金、国内外旅費、会議費、印刷費)
		\end{enumerate}
	\item 研究部門に係る経費
		\begin{enumerate}
			\item 共通的に使用される物品等に係る経費(備品購入費、消耗品費、機器借料、雑役務費、通信運搬費、謝金、国内外旅費、会議費、印刷費、新聞・雑誌代、光熱水費)
			\item 当該研究の応用等による研究活動の推進に係る必要経費(研究者・研究支援者等の人件費、備品購入費、消耗品費、機器借料、雑役務費、通信運搬費、謝金、国内外旅費、会議費、印刷費、新聞・雑誌代、光熱水費)
			\item 特許関連経費
			\item 研究棟の整備、維持及び運営経費
			\item 実験動物管理施設の整備、維持及び運営経費
			\item 研究者交流施設の整備、維持及び運営経費
			\item 設備の整備、維持及び運営経費
			\item ネットワークの整備、維持及び運営経費
			\item 大型計算機(スパコンを含む)の整備、維持及び運営経費
			\item 大型計算機棟の整備、維持及び運営経費
			\item 図書館の整備、維持及び運営経費
			\item ほ場の整備、維持及び運営経費
		\end{enumerate}
	\item その他の関連する事業部門に係る経費
		\begin{enumerate}
			\item 研究成果展開事業に係る経費
			\item 広報事業に係る経費 
		\end{enumerate}
\end{enumerate}

\article{間接経費の執行}
間接経費は、最高管理責任者の責任のもと、総務部が管理を行うこととし、共通指針及び本方針の主な使途を参考として、最高管理責任者及び統括管理責任者の決裁によって適正に執行するものとする。

\article{間接経費の実績報告}
最高管理責任者のもと、間接経費に係る証拠書類を適切に保管した上で、当該競争的資金元の配分機関に対し、毎年度の間接経費使用実績等を共通指針に定められた期限及び様式で報告を行うものとする。

\end{document}
