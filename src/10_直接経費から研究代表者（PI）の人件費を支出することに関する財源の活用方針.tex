\documentclass[10pt,a4paper,uplatex]{jsarticle}
\usepackage{bm}
\usepackage{graphicx}
\usepackage[truedimen,left=25truemm,right=25truemm,top=25truemm,bottom=25truemm]{geometry}
\usepackage{array}
\usepackage{titlesec}
\usepackage{jpdoc}
\usepackage[nomarkers]{endfloat}

\titleformat*{\section}{\large\bfseries}
\def\title{競争的研究費の直接経費から研究代表者(PI)の人件費を支出することに関する財源の活用方針}

\def\alias#1{\十干{#1}}

\begin{document}
	\newpage
{\centering \Large\bf \title  \vskip 0em}
\vskip 2em

\rightline{令和5年5月15日制定}

株式会社RICOS(以下、「当会社」とする。)は、「競争的研究費の直接経費から研究代表者(PI)の人件費の支出について」(令和2年10月9日 競争的研究費に関する関係府省連絡会申し合わせ)に基づき、当会社における競争的研究費の直接経費から研究代表者(以下、「PI」とする。)の人件費を支出することにより確保された財源の活用方針(以下「本方針」という。)について、以下のとおり定めるものとする。

\artivle{対象となる事業}
本方針の取り扱いの対象となる事業は、次のうちその間接経費が直接経費の30%以上のものとする。
\begin{enumerate}
	\item 競争的研究費のうち、資金配分機関が指定するもの
	\item 受託・共同研究等による研究担当者の充当経費が認められたもの
\end{enumerate}

\article{目標}
本方針は、当会社の事業である「計算科学、計算機科学及び応用力学の分野における研究並びに知的財産の創出」および研究成果を社会に公表・還元することにより、人類の進歩と社会の発展に寄与するため、競争的研究費を獲得した PI 等の研究者に対する処遇改善及び優秀な研究者の持続的確保など、研究環境の整備・強化などを図るとともに、当会社の研究力強化に資することを目標とする。

\article{当該目標を達成するための具体的な経費の使途・活用策}
PI の希望に基づき、確保された財源については、以下のために活用するものとする。
\begin{enumerate}
	\item 直接経費から人件費を支出した PI 等の研究者に対する研究環境の改善
	\item 研究者の新規雇用、人材育成のための支援
\end{enumerate}

\article{執行にあたる留意事項等}
直接経費の使途は、当該研究費を獲得した研究者が、自らの責任において研究の着実な遂行のため支出するものであり、当会社が PI の人件費の支出を強制するものではない。
\term 統括管理責任者または部門の長は、競争的研究費および受託・共同研究等で獲得した PI 等の研究者が研究活動に専念し、研究活動が確実に実施できるよう、研究時間の確保に努めなければならない。
\term 本方針については、当会社の研究者の意向および本制度の適用状況を踏まえつつ、必要に応じて見直しを行うものとする。

\article{附則}
本方針に定めるもののほか、PI 人件費の支出に関して必要な事項は、別に定める。

\end{document}