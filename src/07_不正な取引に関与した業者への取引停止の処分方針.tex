\documentclass[10pt,a4paper,uplatex]{jsarticle}
\usepackage{bm}
% \usepackage{graphicx}
\usepackage[dvipdfmx]{graphicx}
\usepackage[truedimen,left=25truemm,right=25truemm,top=25truemm,bottom=25truemm]{geometry}
\usepackage{array}
\usepackage{titlesec}
\usepackage{jpdoc}
\usepackage{tree}
\usepackage[nofiglist,notablist,nomarkers]{endfloat}

\renewcommand{\figurename}{別図}
\newcommand{\figref}[1]{別図\ref{#1}}
\renewcommand{\tablename}{別表}
\newcommand{\tabref}[1]{別表\ref{#1}}

\titleformat*{\section}{\large\bfseries}
\def\title{株式会社科学計算総合研究所\\不正な取引に関与した業者への取引停止の処分方針}

\def\alias#1{\十干{#1}}

\begin{document}
	\newpage
{\centering \Large\bf \title  \vskip 0em}
\vskip 2em

\rightline{令和2年10月5日制定}

\article{目的}
この方針は、「研究機関における公的研究費の管理・監査のガイドライン(実施基準)」(平成19年2月15日(平成26年2月18日改正)文部科学大臣決定)において要請されている事項を踏まえ、株式会社科学計算総合研究所(以下、「当会社」という。)における不正な対応を行った取引業者に対する処分方針を定めることを目的とする。

\article{不正取引及び不正行為の認定}最高管理責任者は、次の各号に定める不正取引に関与した業者について、取引停止等の措置をとるものとする。
\begin{enumerate}
	\item 物品購入、業務委託等に関する提出書類に虚偽の記載があり、契約の相手方として不適当と認められるとき。
	\item 見積書・契約書等に定められた品質・数量について不正行為を行ったと認められるとき、又は業務委託について粗雑な履行を行ったと認められるとき。
	\item 物品購入、業務委託等に関する契約に違反する等、契約の相手方として不適当と認められるとき。
	\item 当会社の職員に対する贈賄が発覚したとき。
	\item 前各号のほか、業務遂行にあたり、不誠実な行為を行い、契約の相手方として不適当と認められるとき。
\end{enumerate}

\article{不正な取引に関与した業者への取引停止等の処分}
取引停止期間は、1ヶ月以上12ヶ月以内とする。ただし、即時の取引停止により当会社の研究活動に著しく影響が生じる場合は、一定期間を経た後に取引停止処分とすることができる。
\term 取引停止措置を受けた業者が、その取引停止措置の期間満了後1年を経過するまでの間に新たな事案により取引停止措置をする場合の期間については、2ヶ月以上24ヶ月以内とする。
\term 不正な取引に関与した業者への具体的な取引停止期間は、当会社の最高管理責任者が状況調査のうえ合理的判断により決定する。

\article{不正防止に向けた取り組み}
未然に不正を防止するために、以下の取り組みを行う。
\begin{enumerate}
	\item 不正な取引に関与した業者への取引停止等を行う。
	\item 取引業者に対し、処分方針および不正対策を周知徹底する。
	\item 一定の取引実績のある取引業者に対し、不正を行わない旨の誓約書の提出を求める。
\end{enumerate}

\article{取引業者に提出を求める誓約書}
取引業者に対し、不正取引を防止するため、機関におけるリスク要因・実効性等を考慮した上で、別途定める様式に示す誓約書の提出を求める。誓約書には以下の内容を含めるものとする。
\begin{enumerate}
	\item 当会社の規程等を遵守し、不正に関与しないこと。
	\item 内部監査、その他調査等において、取引帳簿の閲覧・提出等の要請に協力すること。
	\item 不正が認められた場合は、取引停止を含むいかなる処分を講じられても異議がないこと。
	\item 財団の職員等から不正な行為の依頼等があった場合には通報すること。
\end{enumerate}
\term 一定の取引実績(年間を通じた取引額が100万円を超える)のある取引業者に対し、前項の内容を含んだ誓約書の提出を原則として年1回求めることとする。なお、取引業者の体制等に変更があった場合は再度提出を求めることとする。

\end{document}
