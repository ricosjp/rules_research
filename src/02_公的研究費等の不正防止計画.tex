\documentclass[10pt,a4paper,uplatex]{jsarticle}
\usepackage{bm}
% \usepackage{graphicx}
\usepackage[dvipdfmx]{graphicx}
\usepackage[truedimen,left=25truemm,right=25truemm,top=25truemm,bottom=25truemm]{geometry}
\usepackage{array}
\usepackage{titlesec}
\usepackage{jpdoc}
\usepackage{tree}
\usepackage[nofiglist,notablist,nomarkers]{endfloat}

\renewcommand{\figurename}{別図}
\newcommand{\figref}[1]{別図\ref{#1}}
\renewcommand{\tablename}{別表}
\newcommand{\tabref}[1]{別表\ref{#1}}

\titleformat*{\section}{\large\bfseries}
\def\title{公的研究費等の不正防止計画}

\def\alias#1{\十干{#1}}

\begin{document}
	\newpage
{\centering \Large\bf \title  \vskip 0em}
\vskip 2em

\rightline{令和2年10月5日制定}

(目的)\\
この計画は、「研究機関における公的研究費の管理・監査のガイドライン(実施基準)」(平成19年2月15日(平成26年2月18日改正)文部科学大臣決定)に基づき、株式会社科学計算総合研究所(以下、「当会社」という。)における、公的研究費を活用した研究活動の不正行為を防止し、公的研究費の適正な管理・運営を行うため、次のとおり、不正防止に関する計画を策定する。

\begin{table}[h]
\centering
\scalebox{0.90}{
\begin{tabular}{|l|p{5cm}|p{6cm}|}
\hline
区分 & 不正発生リスク要因 & 防止計画 \\
\hline
1. 責任体制の明確化 & ・責任体制が不明確のため、不正防止対策や不祥事発生時に的確かつ迅速な対策が実施できない。 & ・当会社内における最高管理責任者と統括管理責任者の責任範囲・権限について、説明会等や外部への公開等により、その周知徹底を図る。\\
\hline
\multirow{2}{*}{\shortstack{2. 公的研究費の適正な運
営・\\\ 管理のための基盤の整備}}& ・公的研究費の使用ルールが遵守されていない。& ・使用ルールについての規程を作成し、当会社の研究活動に関わるすべての職員を対象に説明会を実施するとともに、ホームページへの公開や相談窓口の設置により、使用ルールの周知徹底を図る。 \\
& ・公的研究費の使用ルールと実態が乖離している。& ・職員全員が遵守すべき「研究実施規程」「研究活動における不正防止に関する規程」等を定め、研究倫理教育、説明会等により、その周知徹底を図るとともに、意識の向上を図る。\\
\hline
\multirow{4}{*}{\raisebox{30pt}{\shortstack{3. 研究費の適正な運営・管理\\(予算、発注、検収)}}} &  ・予算執行状況が適切に把握されていないため、適切な執行が行
えない。& ・研究計画に基づき、計画的な予算執行の有無を総務部の事務担当者が適
宜確認を行うとともに、必要に応じて改善を求める。 正当な理由により、研究費の執行が当初計画より遅れる場合等においては、繰越制度の積極的活用等、ルールそのものが内蔵する弾力性を利用した対応を行う。
また、研究費を年度内に使い切れずに返還しても、その後の採択等に悪影響はないことを周知徹底する。\\
& ・研究者が発注して総務部の事務担当者が関与していない。& ・発注は原則、総務部の事務担当者が実施する。またやむを得ず、研究者自身による発注を認める場合であっても、可能な範囲をルールで定め明確化する。\\
& ・研究者が検収作業を実施しているなど、納品検収が適切に行われていない。& ・原則、総務部の事務担当者がすべての検収を実施して、納品事実の確認を徹底する。\\
& ・合理性のある理由がないにも関わらず一部の業者に発注が集中するなど、研究者と取引業者が必要以上に密接な関係を持った状態となっている。& ・一定以上の取引業者には不正な取引をしない旨の誓約書の提出を求め、不正な取引に関与した業者に対しては、取引停止等の必要な措
置を講じる。\\
\hline
\multirow{2}{*}{\raisebox{8pt}{4. 情報伝達方法の健全化}} & ・不正使用を発見したにもかかわらず、告発先がわからない。
& ・公益通報、研究活動の不正行為等にかかわる通報の通報先について、説明会等やホームページ上での公開等により、その周知徹底を図る。\\
&・不正使用を発見した者が不利益を恐れて通報・告発を躊躇する。 & ・公益情報、研究活動の不正行為等にかかわる通報・告発に際して、通報者が不利益を受けないことを規程において明確に定めるとともに、説明会等を通じてこのことの周知徹底を図る。 \\
\hline
5. モニタリングの実施 & ・期間の経過に伴い、策定された不正防止計画が陳腐化する。 & ・定期的に不正の発生要因を分析するとともに、点検・評価を行い、必要があれば管理・監査体制や不正防止計画の見直しを実施する。\\
\hline
\end{tabular}}
\label{tab:fuseiboushikeikaku}
\end{table}

\end{document}
