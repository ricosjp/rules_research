\documentclass[10pt,a4paper,uplatex]{jsarticle}
\usepackage{bm}
% \usepackage{graphicx}
\usepackage[dvipdfmx]{graphicx}
\usepackage[truedimen,left=25truemm,right=25truemm,top=25truemm,bottom=25truemm]{geometry}
\usepackage{array}
\usepackage{titlesec}
\usepackage{jpdoc}
\usepackage{tree}
\usepackage[nofiglist,notablist,nomarkers]{endfloat}

\renewcommand{\figurename}{別図}
\newcommand{\figref}[1]{別図\ref{#1}}
\renewcommand{\tablename}{別表}
\newcommand{\tabref}[1]{別表\ref{#1}}

\titleformat*{\section}{\large\bfseries}
\def\title{研究補助者取扱規程}

\def\alias#1{\十干{#1}}

\begin{document}
	\newpage
{\centering \Large\bf \title  \vskip 0em}
\vskip 2em

\rightline{令和3年6月1日制定}

\article{目的}
この規程は、科学計算総合研究所(以下「当会社」という。)このにおける、特定の研究課題等の遂行に必要な技能・技術面での支援を確保することにより、研究課題等を効果的にするとともに、学術研究を支えるための研究支援体制の強化を図ることを目的として、研究補助者の雇用等に必要な事項を定めるものとする。

\article{職務内容}
研究補助者とは、一定の期間、当会社に所属する職員が研究代表者の指示の下に、当該研究課題等支援の業務に従事する。

\article{資格・選考}
研究補助者は、学士の学位を有する者及びこれと同等以上の能力を有すると最高管理責任者が認める者とする。
\term 研究補助者を選考する場合は、当該研究課題の研究代表者からの書面による申請に基づき、最高管理責任者と基盤研究部部長がその雇用についての審査・選考を行うものとする。

\article{所属}
研究補助者の所属は,研究補助者として参画する当該研究課題等を実施する部局等とする。

\article{勤務条件}
当会社の職員を研究補助者として雇用する場合には,当該職員の他の業務等に支障を及ぼすことがないよう配慮するとともに、勤務形態の明確化及び勤務時間の適正管理に留意しなければならない。
\term この規程に定めるもののほか、研究補助者の給与、通勤手当、勤務時間、およびその他就業等に関しては、「株式会社科学計算総合研究所 就業規則」に基づき、それぞれの雇用契約において定める。

\article{運用}
研究補助者に研究補助業務を行わせるに当たっては、本制度の目的を踏まえた具体的運用に係る実施細目を策定するものとする。

\article{補則}
この規程に定めるもののほか、研究補助者の取扱いについて必要な事項は、当該研究の実施要領および受託契約等に基づき、最高管理責任者が定める。

\end{document}
