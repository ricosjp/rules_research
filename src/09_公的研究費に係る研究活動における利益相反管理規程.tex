\documentclass[10pt,a4paper,uplatex]{jsarticle}
\usepackage{bm}
\usepackage{graphicx}
\usepackage[truedimen,left=25truemm,right=25truemm,top=25truemm,bottom=25truemm]{geometry}
\usepackage{array}
\usepackage{titlesec}
\usepackage{jpdoc}
\usepackage[nomarkers]{endfloat}

\titleformat*{\section}{\large\bfseries}
\def\title{公的研究費に係る研究活動における利益相反管理規程}

\def\alias#1{\十干{#1}}

\begin{document}
	\newpage
{\centering \Large\bf \title  \vskip 0em}
\vskip 2em

\article{目的}
株式会社科学計算総合研究所(以下、「当会社」とする)では、研究成果を社会に公表、還元することにより、人類の進歩と社会の発展に寄与する研究機関を目指しており、産業上有意義な知見については人類共通の知的財産として社会に還元することで、産業の振興に貢献することを重要な使命としている。その研究活動の中で、立場上追求すべき利益・目的と、その対象者が他にも有している立場や個人としての利益とが、競合ないしは相反している状態が起こり得る。
本規程の目的は、利益相反となる可能性のある状態に適切に対応するため、利益相反に関する基本的方針を策定することである。

\article{適用範囲}
\label{適用範囲} 
本規程は、当会社及びその子会社の取締役、従業員(正社員及び契約社員)及び臨時従業員(以下、「職員等」とする)に適用される。 また、当会社の関連会社の取締役及び従業員、さらには当会社グループの業務委託を請負う第三者(請負業者、仕入先、仕入先)においても、本規程の理解と遵守を求めていくこととする。

\term 対象とする研究活動の範囲については、以下のとおりとする。
\begin{enumerate}
	\item 各府省、独立行政法人、地方公共団体等から配分される競争的研究資金等、又はそれ以外の資金配分機関から、当該資金の使用及び管理について本規程を準用すべき旨の申し出があった研究資金(以下、「公的研究費」とする)支給を受ける研究
	\item その他最高管理責任者が指定する研究
\end{enumerate}

\article{利益相反の定義}
\label{利益相反の定義} 
本規程では、次の各号に定める状態にあることを利益相反と定義する。
\begin{enumerate}
	\item 外部や当会社における経済的な利益関係等によって、研究活動で必要とされる公正かつ適正な判断が損なわれる、又は損なわれるのではないかと第三者から懸念が表明されかねない状態(狭義の利益相反)
	\item 兼業活動により複数の職務遂行責任が存在することにより、当社の職務における判断が損なわれたり、職務を怠った状態になっている、又はそのような状態にあると第三者から懸念が表明されかねない状態(責務相反)
	\item その他、発生する利害関係の管理を適切に行われなければ、当会社における運営又は研究開発等の個別の業務において当会社との公平性又は中立性が損なわれる可能性があり、利益相反管理の対象とする必要があると認められる状態
\end{enumerate}

\article{利益相反の管理体制}
当会社では、次の各号に定める体制に従い、利益相反の管理を実施する。
\begin{enumerate}
	\item 最高管理責任者は、当社職員等で組織した利益相反管理委員会を設置し、利益相反マネジメントの企画、運用等について審議する。
	\item 当会社は、職員等からの利益相反に関する相談等に適切に対応するため、委員会事務局(以下、「事務局」とする)を窓口に設け対応する。
	\item 本規程に基づき、職員等の自己申告の内容に接することができる者(申告者の所属長、利益相反管理委員会及び事務局)は、利益相反マネジメントにより知り得た職員等の個人情報を適切に管理することとし、管理職務上知り得た秘密を他に漏らしてはならない。その職務を退いた後も同様とする。
\end{enumerate}

\article{利益相反マネジメントの手続き及び方法}
当会社では、次の各項に定める手続きをもって、利益相反の管理を実施することとする。
\term 利益相反に関する自己申告書(以下、「自己申告書」とする)について、次の各号に従い提出する。
\begin{enumerate}
	 \item \ref{適用範囲}及び\ref{利益相反の定義}に該当する職員等は、1年に1回自己申告書を事務局に提出するものとする。
	 \item  前号の規程に関わらず、職員等は、提出した自己申告書の申告内容に変更が生じた場合、又は生じる可能性があると判断した場合、速やかに自己申告書を事務局に提出するものとする。
	 \item 事務局は毎年3月に自己申告書の提出を職員等に呼びかけるものとする。
	 \item 自己申告書の内容は別途定めるものとする。
\end{enumerate}

\term 前項に基づき提出された自己申告書について、次の各号に従い、委員会による確認を実施する。
\begin{enumerate}
	\item 委員会は、事務局に提出された自己申告書について、利益相反による弊害の有無を確認する。
	\item 自己申告書を提出した職員等に対し、必要があると委員会が認めるときは、委員会によるヒアリングを実施し、利益相反による弊害の有無を確認する。
	\item 委員会は、自己申告書を提出した職員等に対し、確認結果を通知する。
\end{enumerate}

\term 委員会は、前項に定める確認の結果、利益相反による弊害が生じている状態にある、又は今後その状態になる可能性があると判断した場合は、当該自己申告書を提出した職員等に対し、兼業活動等の体制の是正、改善又は中止の勧告を行い、勧告に係る措置に関する報告を求める。
\term 自己申告書を提出した職員等は、委員会の勧告に異議があるときは、当該委員会に対して再審査を請求することができる。最高管理責任者は、委員会の審議結果及び当該職員等からの請求の内容を踏まえ最終判定を行い、委員会及び当該職員等に対して、最終判定に基づく措置を命ずる。
\term 当会社は、職員等に対して、利益相反の管理の重要性の周知、利益相反への適切な対処方法(例えば、対外的な場での利益相反の開示・説明等)について必要な研修を行う。

\article{情報の開示}
当会社は本規程を公開するとともに、本規程に従って実施した職員等の利益相反の開示状況について、個人のプライバシーに係る部分を除き内外に開示する。

\article{本規程の見直し}
利益相反の管理全般の動向並びに研究開発等の業務及び当会社の運営に係る実態の変化に応じて、本規程の見直しを適宜実施する。

\article{補則}
この規程に定めるもののほか、利益相反の管理について必要な事項は、最高管理責任者が定める。

\vspace{1cm}
\subparagraph{附則}
この規程は、令和3年10月1日から施行する。

\end{document}