\documentclass[10pt,a4paper,uplatex]{jsarticle}
\usepackage{bm}
\usepackage{graphicx}
\usepackage[truedimen,left=25truemm,right=25truemm,top=25truemm,bottom=25truemm]{geometry}
\usepackage{array}
\usepackage{titlesec}
\usepackage{jpdoc}
\usepackage[nomarkers]{endfloat}

\titleformat*{\section}{\large\bfseries}
\def\title{科学研究費助成事業-科研費-の研究実施規程}

\def\alias#1{\十干{#1}}

\begin{document}
\newpage
{\centering \Large\bf \title  \vskip 0em}
\vskip 2em

\article{目的}
本規程は、株式会社RICOS(以下、「当会社」とする)の研究者が行う研究のうち、科研費を受けて行う研究について、その取扱いの方針を定め、もって科研費による研究成果をあげるとともに研究成果の普及をはかることを目的とする。

\article{組織の責任体制}
組織全体を統括し、科研費の運営・管理について最終責任を負う者(最高管理責任者)を代表取締役と定める。
\term 最高管理責任者を補佐し、科研費の運営・管理について機関全体を統括する実質的な責任と権限を持つ者(統括管理責任者)を基盤研究部長と定める。
\term 科研費の運営・管理について実質的な責任と権限を持つ者(コンプライアンス推進責任者)を基盤研究部長と定める。
\term 研究倫理教育責任者を基盤研究部長と定める。

\article{組織、研究を行う職}
研究活動を行うことを職務に含む者として所属し、研究活動に実際に従事するものは下のとおりである。\\
基盤研究部(部長、主任研究員、研究員)

\article{研究計画の策定}
研究者は、科研費による研究については、他の業務に支障を及ぼさない範囲内において自発的に研究計画を立案し、実施するものとする。
\term 当該研究計画を立案し実施しようとする研究者は、あらかじめ、文部科学省又は独立行政法人日本学術振興会が定める様式に従った研究計画調書を作成し、当該調書の写しを代表取締役に提出するものとする。

\article{研究の実施}
研究者は、科研費による研究を行う場合は、当会社の活動として実施するものとする。

\article{研究成果の取扱い}
研究者は、科研費により行った前条の研究については、他の規程に係わらず、当該研究の研究成果について自らの判断で公表することができるものとする。また、公表に当たっては、職務として自発的に学会等に参加できるものとする。

\article{研究報告の義務}
科研費による研究を行う研究者は、科研費に係る規程及び交付の際に附される諸条件に従い報告書を作成し、当該報告書等の写しを代表取締役に提出するものとする。

\article{管理等の事務}
科研費の研究計画調書の取りまとめおよび補助金の経理管理等の事務は、人事総務部が所掌する。

%(別途、研究費の取扱規程となり得る既存の規程がない場合に、以下の条文を追加し、本規程において定めても構わない。)

\term 人事総務部は、研究者の依頼に基づいて物品の発注を行う。研究者本人は発注を行わない。 
\term 人事総務部は、業者が事務局に持ち込んだ物品について、品名・数量等を確認して検収し、研究室に納品させる。 
\term 人事総務部は、研究者の依頼に基づいて出張伺いの決裁を取る。用務終了後に、出張報告書、領収書及び航空券半券等により事実確認を行う。 
\term 人事総務部は、研究者からの依頼に基づいて職員の雇用伺いの決裁を取る。作業終了後に勤務報告等により、事実確認を行う。 

\article{内部監査}
研究費の適正な管理のため、「研究機関における公的研究費の管理・監査のガイドライン(実施基準)」(平成19年2月15日(平成26年2月18日改正)文部科学大臣決定)を踏まえ、内部監査を行う。
\term 内部監査は、人事総務部が行う。
\term 監査の対象は、前年度の契約実績の約10%を抽出したものとし、会計書類の検査並びに購入物品の使用状況等に関する研究者からのヒアリングにより確認する。

\article{コンプライアンス教育等}
「研究機関における公的研究費の管理・監査のガイドライン(実施基準)」(平成19年2月15日(平成26年2月18日改正)文部科学大臣決定)を踏まえ、科研費の運営・管理に関わる全ての構成員にコンプライアンス教育を行い、不正を行わないことなどを盛り込んだ誓約書を提出させる。誓約書の提出がない場合は、科研費の管理・運営に関わらせない。
\term 公正な研究活動を推進するため、「研究活動における不正行為への対応等に関するガイドライン」(平成26年8月26日文部科学大臣決定)を踏まえ、研究活動に関わる者を対象に定期的に研究倫理教育を行う。

\article{法令等の遵守}
当会社に所属する研究者は科研費による研究の遂行に当たり、関係法令等並びに文部科学省及び独立行政法人日本学術振興会が定める各種の科研費に関するルールを遵守するものとする。
\\

\subparagraph{附則}本規程は、令和2年7月1日から施行する。
\subparagraph{附則}本規程の改正は、令和5年11月1日から施行する。

\end{document}
