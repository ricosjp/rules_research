\documentclass[10pt,a4paper,uplatex]{jsarticle}
\usepackage{bm}
% \usepackage{graphicx}
\usepackage[dvipdfmx]{graphicx}
\usepackage[truedimen,left=25truemm,right=25truemm,top=25truemm,bottom=25truemm]{geometry}
\usepackage{array}
\usepackage{titlesec}
\usepackage{jpdoc}
\usepackage{tree}
\usepackage[nofiglist,notablist,nomarkers]{endfloat}

\renewcommand{\figurename}{別図}
\newcommand{\figref}[1]{別図\ref{#1}}
\renewcommand{\tablename}{別表}
\newcommand{\tabref}[1]{別表\ref{#1}}

\titleformat*{\section}{\large\bfseries}
\def\title{株式会社科学計算総合研究所\\公的研究費等の内部監査に関する規程}

\def\alias#1{\十干{#1}}

\begin{document}
	\newpage
{\centering \Large\bf \title  \vskip 0em}
\vskip 2em

\article{目的}
この規程は、「研究機関における公的研究費の管理・監査のガイドライン(実施基準)」(平成19年2月15日(平成26年2月18日改正)文部科学大臣決定)において要請されている事項を踏まえ、株式会社科学計算総合研究所(以下、「当会社」という。)における内部監査の制度、実施及び報告に関する基本的事項と監査を実施するための手順を定める。

%公的研究費等の内部監査要項

\article{内部監査部門}
内部監査部門は、最高管理責任者である代表取締役の直轄的な組織とする。

\article{監査担当者}
公的研究費等のモニタリング及び監査を行うために、最高管理責任者は内部監査部門を設置する。内部監査部門は次に掲げる者をもって組織する。
\begin{itemize}
	\item 最高管理責任者
	\item 統括管理責任者
	\item コンプライアンス推進責任者
	\item 監査担当者として最高管理責任者が必要と認める総務部の職員
\end{itemize}

\article{監査の種類}
監査の種類は、次に定めるものとする。
\begin{enumerate}
    \item 通常監査\\
    公的研究費等の研究課題における遂行状況及び経費の執行状況について実施する監査をいう。
    \item リスクアプローチ監査\\
    「公的研究費等に関する不正防止計画」に定める「不正発生リスク要因」を参考にするなど、不正が発生するリスクに対して、重点的にサンプルを抽出し、抜き打ち等を含めて実施する監査をいう。
\end{enumerate}

\article{監査の実施の時期}内部監査は、年1回適時実施するものとする。また、不正発生件数が多いと思われる場合には、必要に応じて、リスクアプローチ監査を実施する。

\article{監査の対象}
監査の対象は、前年度の公的研究費等の資金に係る業務全般とする。

\article{通常監査事項}
通常監査を行う年度において、当会社に所属する研究者が研究代表者として科学研究費助成事業を除く公的研究費の交付を受けている研究課題数の概ね10\%以上を対象とする。科学研究費助成事業については、別途定めている「科学研究費助成事業-科研費-の研究実施規程」に従うものとする。
\begin{itemize}
	\item 収支簿
	\item 証拠書類(見積書、請求書、納品書、領収書等)
	\item 固定資産(備品等)の納品検収記録
	\item 謝金関係資料
	\item 旅費関係資料
	\item その他、監査に係る必要な事項
\end{itemize}

\article{リスクアプローチ監査事項}
リスクアプローチ監査は、次に掲げる項目について行う。
\begin{itemize}
	\item 研究者の旅費の一定期間分抽出による出張を対象とした概要(目的、内容、交通手段、宿泊場所など)に関するヒアリング
	\item 非常勤雇用者を対象とした勤務実態(勤務内容、勤務時間など)に関するヒアリング
	\item 内部監査部門で協議し監査対象とされた購入物品につき、研究目的との整合性、使用状況に関するヒアリング及び現物確認
	\item 予算執行が研究計画に比して遅れている研究者へのヒアリング
	\item 一定の取引実績のある取引業者の帳簿との突合で、架空発注がないかの確認
\end{itemize}

\article{不正防止計画の実施状況確認}
内部監査部門は、内部監査実施に合わせ、不正防止計画をはじめとする当会社全体の具体的な対策の実施状況を確認する。

\article{監査担当者の権限}
監査担当者の権限は、次のとおりとする。
\term 被監査部門の関係者に対し、帳票や諸資料の提出並びに事実の説明、その他監査実施上必要なもの等を求めることができる。
\term 監査実施上必要と認められる各種会議への出席または議事録の閲覧を求めることができる。

\article{被監査部門の義務}
被監査部門は、円滑かつ効果的に監査が実施できるよう積極的に協力しなければならない。

\article{監査担当者の義務}
監査担当者は、次の事項を遵守しなければならない。
\begin{itemize}
	\item 監査担当者は、業務上知り得た事項は、正当な理由なくして他に漏えいしてはならない。
	\item 監査は、事実に基づいて行い、常に公正に判断しなければならない。
	\item 監査担当者は、いかなる場合においても被監査部門の業務の処理・方法等について、直接指揮命令をしてはならない。
\end{itemize}

\article{監査の実施}
監査担当者は、監査の実施にあたり、予め監査日時・対象者について最高管理責任者に承認を得るものとする。

\article{監査の通知}
監査担当者は、監査の実施にあたり、予め監査日程とともに監査対象者へ通知するものとする。ただし、緊急または特に必要があると認める場合は、事前に通知することなく監査を実施することができる。

\article{監査結果の報告等}
監査担当者は、実施した監査について監査結果報告書を作成し、最高管理責任者に提出するものとする。

\article{改善是正の措置}
最高管理責任者は、改善または是正の必要があるものについては、基盤研究部長または主任研究員等を通してその措置を命ずるものとする。措置を命ぜられた長等は、直ちにその措置をとり、最高管理責任者に報告しなければならない。

\article{監事との連携}
内部監査部門は、前項の改善是正の措置も含め監査結果等について、監事に報告し、意見を求めるなど連携し、今後の監査計画策定及び監査実施に資するものとする。

\article{結果報告の取扱い}
監査報告の取りまとめ結果については、コンプライアンス教育の一環として、当会社内で周知を図り、類似事例の再発防止を徹底するものとする。

\article{補則}
この規程に定めるもののほか、監査の実施について必要な事項は、最高管理責任者が定める。

\vspace{1cm}
\subparagraph{附則}
この規程は、令和2年10月5日から施行する。

\end{document}
